\begin{problem}{Stones}{standard input}{standard output}{3 seconds}

Alicia and Roberto like to play games. Today, they are playing a game where there's a pile of stones on the table. The two players alternate turns removing some stones from the pile. On the first turn, a player can remove any number of stones but he or she must leave at least one stone on the table. On following turns, a player can remove at most twice the number of stones that the other player removed on the previous turn. Each player must always remove at least one stone.

The player that takes the last stone wins.

For example, let's imagine there's a pile of 8 stones on the table, and let's imagine Alicia starts.

\begin{itemize}
  \item{On the first turn, Alicia must remove a number of stones between 1 and 7 (according to the rules above, on the first turn she can remove any number of stones as long as there remains at least 1). Let's say Alicia takes 2 stones, leaving 6 on the table.}
  \item{On the second turn, Roberto must remove at least 1 stone and at most 4 stones (because 4 is twice the number of stones that Alicia removed in the previous turn). Let's say he takes 3, leaving 3 stones on the table.}
  \item{On the third turn, Alicia must remove at least 1 and at most 6 stones (6 is twice the number of stones Roberto took in the previous turn). Since only 3 stones remain on the table, she simply takes the 3 stones and wins the game!}
\end{itemize}

However, Roberto could have played better. If he had taken only 1 stone on the second turn, he would have left 5 on the table with Alicia having to take between 1 and 2 stones. If Alicia took 2, Roberto could win immediately by taking the 3 stones that would remain. So Alicia's only choice is to take 1 stone, leaving 4 stones and Roberto having to take between 1 and 2 stones. Roberto would then take 1 stone, leaving 3 stones and Alicia having to take between 1 and 2 stones. Roberto could then win after Alicia's move, regardless of whether she takes 1 or 2 stones. In fact, it turns out that Roberto always has a winning strategy if there are 8 stones and Alicia plays first, even if Alicia plays in the best possible way!

Your task is to determine who wins the game if both players play optimally, assuming Alicia always plays first.

\InputFile

The input contains several test cases.

Each test case contains a single integer $n$ $(2 \leq n \leq 1000)$, the number of stones that are initially on the table.

The last line of the input contains a single $0$ and should not be processed.

\OutputFile

For each test case, output \texttt{Alicia} or \texttt{Roberto} on a single line, depending on who wins if both players play optimally and Alicia plays on the first turn.

\Example

\begin{example}
\exmp{
8
2
4
986
987
0
}{
Roberto
Roberto
Alicia
Alicia
Roberto
}%
\end{example}

\end{problem}
