\begin{problem}{Exclusive Access 2}{exclusive.in}{exclusive.out}{2 seconds}

% Author: Roman Elizarov

Having studied mutual exclusion protocols in the previous year's competition you are
now facing a more challenging problem. You have a big enterprise system with a number
concurrently running processes. The system has several resources --- databases, message queues, etc.
Each concurrent process works with two resources at a time. For example, one process might 
copy a job from a particular database into the message queue, the other process might take a job from 
the message queue, perform the job, and then put the result into some other message queue, etc.

All resources are protected from concurrent access by mutual exclusion protocols also known as \emph{locks}. 
For example, to access a particular database process acquires the lock for this
database, then performs its work, then releases the lock. No two processes can hold the same lock
at the same time (that is the property of mutual exclusion). Thus, the process that tries to
acquire a lock \emph{waits} if that lock is taken by some other process.

The main loop of the process that works with resources $P$ and $Q$ looks like this:

\begin{quote}
\begin{verbatim}
loop forever
  DoSomeNonCriticalWork()
  P.lock()
  Q.lock()
  WorkWithResourcesPandQ()
  Q.unlock()
  P.unlock()
end loop
\end{verbatim}
\end{quote}

The order in which locks for resources P and Q are taken is important. Consider a case where process $c$
had acquired lock $P$ with \texttt{P.lock()} and is waiting for lock $Q$ in \texttt{Q.lock()}. 
It means that lock $Q$ is taken by some other 
process $d$. If the process $d$ is working (not waiting), then we say that there is a \emph{wait chain} of
length 1. If $d$ had acquired lock $Q$ and is waiting for another lock $R$, which is acquired by a working process $e$,
then we say that there is a \emph{wait chain} of length 2, etc. If any process in this wait chain waits for
lock $P$ that is already taken by process $c$, then we say that the wait chain has infinite length and the system 
\emph{deadlocks}. 

For this problem, we are interested only in alternating wait chains where processes hold their 
first locks and wait for the second ones. Formally:

\begin{quote}
   \emph{Alternating wait chain} of length $n$ ($n \ge 0$) is an alternating sequence of resources $R_i$ ($0 \le i \le n+1$) and
   distinct processes $c_i$ ($0 \le i \le n$): $R_0$ $c_0$ $R_1$ $c_1$ $...$ $R_n$ $c_n$ $R_{n+1}$, where
   process $c_i$ acquires locks for resources $R_i$ and $R_{i+1}$ in this order.
   Alternating wait chain is a deadlock when $R_0 = R_{n+1}$. 
\end{quote}

You are given a set of resources each process works with. Your task is to decide the order in which each process
has to acquire its resource locks, so that the system never deadlocks and the maximum length of any possible
alternating wait chain is minimized.

\InputFile

The first line of the input file contains a single integer $n$ ($1 \le n \le 100$) --- the number of processes.

The following $n$ lines describe resources that each process needs. Each resource is designated with an
uppercase English letter from \texttt{L} to \texttt{Z}, so there are at most 15 resources. 
Each line describing process contains two different resources separated by a space.

\OutputFile

On first line of the output file write a singe integer number $m$ --- the minimally possible length of the 
maximal alternating wait chain.

Then write $n$ lines --- one line per process. On each line write two resources in the
order they should be taken by the corresponding process to ensure this minimal length of the maximal 
alternating wait chain. Separate resources on a line by a space. If there are multiple satisfying 
orderings, then write any of them. The order of the processes in the output should correspond 
to their order in the input.

\Example

\begin{example}
\exmp{
2
P Q
R S
}{
0
P Q
R S
}%
\exmp{
6
P Q
Q R
R S
S T
T U
U P
}{
0
P Q
R Q
R S
T S
T U
P U
}%
\exmp{
4
P Q
P Q
P Q
P Q
}{
0
P Q
P Q
P Q
P Q
}%
\exmp{
3
P Q
Q R
R P
}{
1
P Q
Q R
P R
}%
\exmp{
6
P Q
Q S
S R
R P
P S
R Q
}{
2
P Q
Q S
R S
P R
P S
R Q
}%
\end{example}

\end{problem}
